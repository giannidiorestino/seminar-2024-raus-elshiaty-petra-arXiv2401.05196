\RequirePackage{etex} 

\documentclass[12pt,a4paper]{amsart}

\usepackage{accents}
\usepackage{cleveref}
\usepackage{fullpage}
\usepackage{graphicx}
\usepackage{macros}
\usepackage{mathtools}
\usepackage{natbib}

\title{Seminar}
\author{Giovanni Pistone}
\date{\today}

\begin{document}

\maketitle

\section{Introduction}

This is an exposition of \citet*{raus|elshiaty|petra:arXiv2401.05196} intended to present the same result and applications in the language on statistical bundles.

\section{Examples}
\label{sec:examples}

\subsection{Poisson}

The Poisson probability function is
\begin{equation}
  \label{eq:1}
  p(y,\eta) = \frac{\eta^y}{y!} \euler^{-\eta} \ , \quad y \in \posintegers \ , \quad \eta \in ]0,+\infty] \ .
\end{equation}

Let us change the parameter to $\theta = \log \eta$, $\theta \in \reals$. Then
\begin{equation}
  \label{eq:2}
  p(y;\eta) = \frac{\euler^{\theta y}}{y!} \euler^{-\eta} = \expof{\theta y - (\euler^{\theta} - 1)} p(y;1) \ ,
\end{equation}
that is, the cumulant function is
\begin{equation}
  \label{eq:3}
  \psi(\theta) = \euler^\theta - 1 \ , \psi'(\theta) = \euler^{\theta} = \eta \ ,
\end{equation}
so that $\theta$ is the natural parameter and $\eta$ is the expectation parameter.

The model is the exponential sub-family of the maximal exponential generated by the affine space $V = \spanof{u,1}$ with $u \colon \posintegers \ni y \mapsto y \in \reals$ (the embedding).

The non-parametric e-chart is
\begin{multline}
  \label{eq:4}
  (\eta_1,\eta_2) \mapsto (p_{\eta_1},p_{\eta_2}) \mapsto \log \frac{\frac{\eta_2^y}{y!}\euler^{-\eta_2}}{\frac{\eta_1^y}{y!}\euler^{-\eta_1}} - \expectat {p_{\eta_1}}{\log \frac{\frac{\eta_2^y}{y!}\euler^{-\eta_2}}{\frac{\eta_1^y}{y!}\euler^{-\eta_1}}} = \\
  y \log \frac{\eta_2}{\eta_1} \euler^{\eta_1-\eta_2} - \expectat {p_{\eta_1}}{y\log \frac{\eta_2}{\eta_1} \euler^{\eta_1-\eta_2}} = \\
  y \log \frac{\eta_2}{\eta_1} \euler^{\eta_1-\eta_2} - \eta_1 \log \frac{\eta_2}{\eta_1} \euler^{\eta_1-\eta_2} = 
  (y - \eta_1) \log \frac{\eta_2}{\eta_1} \euler^{\eta_1-\eta_2} \ .
\end{multline}

In the natural parameter, the e-chart is simply
\begin{equation}
  \label{eq:6}
  (\theta_1,\theta_2) \mapsto (\theta_2-\theta_1) (u - \euler^{\theta_1}) \ .
\end{equation}

The parallelogram law is
\begin{multline}
  \label{eq:7}
  (\theta_2 - \theta_1)(u - \euler^{\theta_1}) + \etransport {\theta_2} {\theta_1} (\theta_3 - \theta_2)(u - \euler^{\theta_2}) = \\  (\theta_2 - \theta_1)(u - \euler^{\theta_1}) + (\theta_3 - \theta_2)(u - \euler^{\theta_1}) = (\theta_3 - \theta_1)(u - \euler^{\theta_1})
\end{multline}

% \bibliographystyle{amsplain}
\bibliographystyle{plainnat}
\bibliography{tutto}

\end{document}

%%% reftex-default-bibliography: ("./")
%%% Local Variables:
%%% mode: latex
%%% TeX-master: t
%%% End:
