\RequirePackage{etex} 

\documentclass[12pt,a4paper]{amsart}

\usepackage{accents}
\usepackage{cleveref}
\usepackage{fullpage}
\usepackage{graphicx}
\usepackage{macros}
\usepackage{mathtools}
\usepackage{natbib}

\title{Seminar}
\author{Giovanni Pistone}
\date{\today}

\begin{document}

\maketitle

\section{Introduction}

This is an exposition of \citet*{raus|elshiaty|petra:arXiv2401.05196} intended to present the same result and applications in the language on statistical bundles according to \citet*{chirco|pistone:2022}.

\section{Examples}
\label{sec:examples}

\subsection{Poisson}

The Poisson probability function is
\begin{equation}
  \label{eq:1}
  p(y,\eta) = \frac{\eta^y}{y!} \euler^{-\eta} \ , \quad y \in \posintegers \ , \quad \eta \in ]0,+\infty] \ .
\end{equation}

Let us change the parameter to $\theta = \log \eta$, $\theta \in \reals$. Then $\euler^\theta = \eta$ and
\begin{equation}
  \label{eq:2}
  p(y;\euler^{\theta}) = \frac{\euler^{\theta y}}{y!} \euler^{-\euler^\theta} = \expof{\theta y - \log y! - \euler\theta} =\expof{\theta y - (\euler^{\theta} - 1)} p(y;1) \ .
\end{equation}
The cumulant function in the reference probability function $p_1$ is
\begin{equation}
  \label{eq:3}
  \psi(\theta) = \euler^\theta - 1 \ , \psi'(\theta) = \euler^{\theta} = \eta \ ,
\end{equation}
so that $\theta$ is the natural parameter and $\eta$ is the expectation parameter.

The model is the exponential family $\expmodat V$ generated by the affine space $V = \spanof{u,1}$, $u \colon \posintegers \ni y \mapsto y \in \reals$, in the maximal exponential family $\maxexpat {p_1}$'s.

The fibre at $p_\eta$ of the statistical bundle is
\begin{equation}
  \label{eq:5}
    \expfibreat {p_\eta} {V} = \spanof{u - \eta} = \setof{\xi(u-\eta)}{\xi \in \reals}
\end{equation}
with covariance form
\begin{equation}
  \label{eq:8}
  (\xi_1(u - \eta),\xi_2(u-\eta)) \mapsto \covat {p_\eta}{\xi_1(u - \eta)}{\xi_2(u-\eta)} = \xi_1\xi_2 \expectat {p_\eta} {(u - \eta)^2} = \xi_1\xi_2 \eta \ .
\end{equation}
Compare with \cref{eq:3}.

In the natural parameter, the exponential dispacement is
\begin{multline}
  \label{eq:6}
  (\theta_1,\theta_2) \mapsto \left(p_{\euler^{\theta_1}},p_{\euler^{\theta_2}}\right) \mapsto S_{p_{\euler^{\theta_1}}}(p_{\euler^{\theta_2}}) = \log \frac {p_{\euler^{\theta_2}}}{p_{\euler^{\theta_1}}} - \expectat {p_{\euler^{\theta_1}}} {\log \frac {p_{\euler^{\theta_2}}}{p_{\euler^{\theta_1}}}} = \\ (\theta_2 - \theta_1) u - (\euler^{\theta_2} - \euler^{\theta_1}) - \expectat {p_{\euler^{\theta_1}}} {(\theta_2 - \theta_1) u - (\euler^{\theta_2} - \euler^{\theta_1})} = \\  (\theta_2-\theta_1) (u - \euler^{\theta_1}) \mapsto \theta_2 - \theta_1 = S_{\theta_1}(\theta_2) \ ,
\end{multline}
while, in the expectation parameter, the exponential displacement is
\begin{equation}
  \label{eq:9}
  (\eta_1,\eta_2) \mapsto (p_{\eta_1},p_{\eta_2}) \mapsto S_{p_{\eta_1}}(p_{\eta_2}) = \log \frac {\eta_2}{\eta_1} (u - \eta_1) \mapsto \log \frac {\eta_2}{\eta_1} = S_{\eta_1}(\eta_2) \ .
\end{equation}

The exponential parallel transport is
\begin{equation}
  \label{eq:10}
  \etransport {p_{\eta_1}} {p_{\eta_2}} \colon \expfibreat {p_{\eta_1}} {V} \ni \xi(u - \eta_1) \mapsto \xi(u - \eta_2) \in \expfibreat {p_{\eta_2}} V \ ,
\end{equation}
that is, is the identity in the $\xi$-coordinates, $\etransport {\eta_1}{\eta_2} \xi = \etransport {\theta_1}{\theta_2} \xi = \xi$.

The parallelogram law for the exponential displacement is 
\begin{equation}
  \label{eq:11}
  S_{p_{\eta_1}}(p_{\eta_2}) + \etransport {p_{\eta_2}}{p_{\eta_1}} S_{p_{\eta_2}}(p_{\eta_3}) = S_{p_{\eta_1}}(p_{\eta_3}) \ ,
\end{equation}
that is,
\begin{equation}
  \label{eq:12}
  \log \frac {\eta_2}{\eta_1}(u - \eta_1) + \log \frac {\eta_3}{\eta_2} \etransport {p_{\eta_2}}{p_{\eta_1}} (u - \eta_2) = \log \frac {\eta_3}{\eta_1}  (u - \eta_1) \ . 
\end{equation}

The velocity of the curve $\eta \mapsto p_\eta$ is
\begin{equation}
  \label{eq:13}
  \derivby \eta \log p_\eta = \derivby \eta (\log \eta u - \eta - log u!) = \frac 1 \eta u - 1  
\end{equation}
and the velocity of the curve $\theta \mapsto p_{\euler^\theta}$ is 
\begin{equation}
  \label{eq:14}
  \derivby \theta \log p_{\euler^\theta} = \derivby \theta (\theta u - \euler^\theta - \log u!) = u - \euler^\theta \ . 
\end{equation}

The curve $t \mapsto p_{\eta(t)}$ is a geodesic with velocity $\xi (u - \eta)$ at $\eta(0) = \eta$ if the velocity is constant up to the transport,
\begin{equation}
  \label{eq:15}
  \derivby t \log p_{\eta(t)} = \frac {\dot\eta(t)} {\eta(t)} u - 1 = \xi (u - \eta(t)) \ ,  
\end{equation}
that is,
\begin{equation}
  \label{eq:16}
  \dot \eta(t) = \xi (u - \eta(t))\eta(t) \ , \quad \eta(0) = \eta \ .
\end{equation}

To solve, consider
\begin{equation}
  \label{eq:17}
  \derivby \eta - \frac 1 u \log \frac {u - \eta} \eta = \frac 1 {(u - \eta) \eta} \ ,
\end{equation}
etc.

Next step is to compute the dual of $\etransport . .$.
% \bibliographystyle{amsplain}
\bibliographystyle{plainnat}
\bibliography{tutto}

\end{document}

%%% reftex-default-bibliography: ("./")
%%% Local Variables:
%%% mode: latex
%%% TeX-master: t
%%% End:
